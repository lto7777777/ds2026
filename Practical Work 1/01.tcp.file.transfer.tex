\documentclass[12pt]{article}

\usepackage[margin=1in]{geometry}
\usepackage{graphicx}
\usepackage{listings}
\usepackage{xcolor}
\usepackage{caption}

\lstset{
  basicstyle=\ttfamily\footnotesize,
  breaklines=true,
  frame=single,
  backgroundcolor=\color{gray!5},
}

\title{Practical Work 1: TCP File Transfer}
\author{Pham Huu Minh   \ID: 23BI14302}
\date{}

\begin{document}
\maketitle

\section{Goal}

The goal of this practical work is to implement a 1--1 file transfer over
TCP/IP using a Command Line Interface (CLI).  
The system is composed of:
\begin{itemize}
    \item one TCP server,
    \item one TCP client,
    \item communication over IPv4 sockets on localhost.
\end{itemize}

The implementation language is C, compiled and executed on Kali Linux
with \texttt{gcc}.

\section{Protocol Design}

The protocol is intentionally simple. The client and server use a fixed
TCP port (8080) and a fixed IP address (127.0.0.1). No application-level
headers are sent:

\begin{itemize}
    \item the client opens file \texttt{example\_file.txt} and sends its
          content as a byte stream;
    \item the server receives bytes on the socket and writes them to
          \texttt{received\_file.txt};
    \item the end of file transfer is detected when the client closes
          the connection and the server's \texttt{recv()} returns~0.
\end{itemize}

Figure~\ref{fig:flow} describes the client--server interaction from
socket creation to file transfer completion.

\begin{figure}[h]
  \centering
  \includegraphics[width=0.65\textwidth]{resource/1.png}
  \caption{Client--server TCP file transfer flow (socket, connect,
  write/read, close).}
  \label{fig:flow}
\end{figure}

At a higher level, the communication can be viewed as the client
connecting to the server through the network, sending the file, and
closing the connection (Figure~\ref{fig:arch-simple}).

\begin{figure}[h]
  \centering
  \includegraphics[width=0.7\textwidth]{resource/2.png}
  \caption{High-level architecture: Client--Network--Server.}
  \label{fig:arch-simple}
\end{figure}

\section{System Organization}

The system is organized into two programs: \texttt{server} and
\texttt{client}. Both run in the same directory
\texttt{/home/kali/Documents/Practical Work/Practical Work 1}.

\subsection*{Server}

The server performs the following steps:
\begin{enumerate}
    \item create a TCP socket (\texttt{socket()});
    \item bind the socket to IP 127.0.0.1 and port 8080 (\texttt{bind()});
    \item listen for incoming connections (\texttt{listen()});
    \item accept a single connection (\texttt{accept()});
    \item receive file data using \texttt{recv()} and write it to
          \texttt{received\_file.txt};
    \item close the connection.
\end{enumerate}

\subsection*{Client}

The client performs these steps:
\begin{enumerate}
    \item create a TCP socket;
    \item connect to 127.0.0.1:8080 (\texttt{connect()});
    \item open \texttt{example\_file.txt} in read mode;
    \item send file content to the server using \texttt{send()};
    \item close the socket when finished.
\end{enumerate}

\section{Implementation Details}

\subsection{Server Code (C)}

\begin{lstlisting}[language=C,caption={Server program (server.c).}]
#include <stdio.h>
#include <stdlib.h>
#include <string.h>
#include <arpa/inet.h>
#include <unistd.h>

#define SIZE 1024

void write_file(int sockfd) {
    int n;
    FILE *fp;
    char *filename = "received_file.txt";
    char buffer[SIZE];

    fp = fopen(filename, "w");
    if (fp == NULL) {
        perror("Error in creating file.");
        exit(1);
    }
    while (1) {
        n = recv(sockfd, buffer, SIZE, 0);
        if (n <= 0) {
            break;
        }
        fprintf(fp, "%s", buffer);
        bzero(buffer, SIZE);
    }
    fclose(fp);
}

int main() {
    char *ip = "127.0.0.1";
    int port = 8080;
    int e;

    int sockfd, new_sock;
    struct sockaddr_in server_addr, new_addr;
    socklen_t addr_size;

    sockfd = socket(AF_INET, SOCK_STREAM, 0);
    if (sockfd < 0) {
        perror("Error in socket");
        exit(1);
    }
    printf("Server socket created.\n");

    server_addr.sin_family = AF_INET;
    server_addr.sin_port = port;
    server_addr.sin_addr.s_addr = inet_addr(ip);

    e = bind(sockfd, (struct sockaddr *)&server_addr, sizeof(server_addr));
    if (e < 0) {
        perror("Error in binding");
        exit(1);
    }
    printf("Binding successfully.\n");

    if (listen(sockfd, 10) == 0) {
        printf("Listening...\n");
    } else {
        perror("Error in listening");
        exit(1);
    }

    addr_size = sizeof(new_addr);
    new_sock = accept(sockfd, (struct sockaddr *)&new_addr, &addr_size);

    write_file(new_sock);
    printf("Data written in the text file.\n");

    close(new_sock);
    close(sockfd);
    return 0;
}
\end{lstlisting}

\subsection{Client Code (C)}

\begin{lstlisting}[language=C,caption={Client program (client.c).}]
#include <stdio.h>
#include <stdlib.h>
#include <unistd.h>
#include <string.h>
#include <arpa/inet.h>

#define SIZE 1024

void send_file(FILE *fp, int sockfd) {
    char data[SIZE] = {0};

    while (fgets(data, SIZE, fp) != NULL) {
        if (send(sockfd, data, sizeof(data), 0) == -1) {
            perror("Error in sending data");
            exit(1);
        }
        bzero(data, SIZE);
    }
}

int main() {
    char *ip = "127.0.0.1";
    int port = 8080;
    int e;

    int sockfd;
    struct sockaddr_in server_addr;
    FILE *fp;
    char *filename = "example_file.txt";

    sockfd = socket(AF_INET, SOCK_STREAM, 0);
    if (sockfd < 0) {
        perror("Error in socket");
        exit(1);
    }
    printf("Server socket created.\n");

    server_addr.sin_family = AF_INET;
    server_addr.sin_port = port;
    server_addr.sin_addr.s_addr = inet_addr(ip);

    e = connect(sockfd, (struct sockaddr *)&server_addr, sizeof(server_addr));
    if (e == -1) {
        perror("Error in connecting");
        exit(1);
    }
    printf("Connected to server.\n");

    fp = fopen(filename, "r");
    if (fp == NULL) {
        perror("Error in reading file.");
        exit(1);
    }

    send_file(fp, sockfd);
    printf("File data sent successfully.\n");

    close(sockfd);
    printf("Disconnected from the server.\n");
    return 0;
}
\end{lstlisting}

\section{Execution and Testing}

All experiments are run on Kali Linux. The source files are compiled
with \texttt{gcc} and the input file is created using the shell:

\begin{lstlisting}[language=bash]
gcc server.c -o server
gcc client.c -o client
echo "Hello from client" > example_file.txt
\end{lstlisting}

Figure~\ref{fig:compile-run} shows the compilation commands and the
creation of \texttt{example\_file.txt}.

\begin{figure}[h]
  \centering
  \includegraphics[width=0.75\textwidth]{resource/6.png}
  \caption{Compiling the C programs and creating the input file.}
  \label{fig:compile-run}
\end{figure}

The server is started in one terminal and the client in another
(Figure~\ref{fig:terminals}). The log messages confirm that the
connection is established and that the file is sent and written
successfully.

\begin{figure}[h]
  \centering
  \includegraphics[width=\textwidth]{resource/3.png}
  \caption{Server and client running in two terminals on Kali Linux.}
  \label{fig:terminals}
\end{figure}

The working directory after execution contains the compiled binaries and
both text files (\texttt{example\_file.txt} and
\texttt{received\_file.txt}), as shown in
Figure~\ref{fig:thunar}.

\begin{figure}[h]
  \centering
  \includegraphics[width=0.6\textwidth]{resource/4.png}
  \caption{Files in the project directory after a successful transfer.}
  \label{fig:thunar}
\end{figure}

Figure~\ref{fig:example} shows the content of \texttt{example\_file.txt}
on the client side.

\begin{figure}[h]
  \centering
  \includegraphics[width=0.6\textwidth]{resource/5.png}
  \caption{Content of \texttt{example\_file.txt} (client side).}
  \label{fig:example}
\end{figure}

Finally, Figure~\ref{fig:route} shows the network configuration
(\texttt{ip r}) confirming that both processes run on the same host.

\begin{figure}[h]
  \centering
  \includegraphics[width=0.8\textwidth]{resource/7.png}
  \caption{IP routing table on Kali Linux used during testing.}
  \label{fig:route}
\end{figure}

\section{Conclusion}

In this practical work, a minimal but functional TCP file transfer
system was implemented in C on Kali Linux. The client and server
programs use the standard BSD socket API to establish a connection over
127.0.0.1:8080 and transfer a text file from client to server.

The tests show that the design and implementation correctly achieve the
goal of one-to-one file transfer over TCP/IP. Possible extensions
include adding support for variable filenames, binary files, error
codes, and multi-client handling.

\end{document}

